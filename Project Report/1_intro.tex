\section{Introduction} 
\label{sec:intro}
\onehalfspacing
\noindent
In the famous paper of Richard Feynman, namely "Simulating Physics with Computer", he pointed out that it is impossible to simulate quantum mechanical systems with classical computers due to their complexity \cite{feynman1982simulating}. He also stated that utilizing these complexities of quantum systems using a quantum computer can have a broad revolutionary impact on many challenging problems. According to the DiVincenzo criteria, qubit is one of the necessary criteria for building a quantum computer \cite{divincenzo2000physical}. Extensive research is going on building a quantum computer with different types of qubits such as neutral atoms, ions, superconducting Josephson junctions and optomechanical resonators.

% \subsection{Neutral Atom Qubit}
Among the numerous categories investigated, arrays of single neutral atoms seem to be a very robust and promising technology \cite{saffman2016quantum}. Its long coherence time and high scalability gave it a distinct edge over other qubits, while achieving high gate fidelity on neutral atoms remain a challenge \cite{xia2015randomized}. In a neutral atom qubit, any two states out of many available states can be used for a qubit; however, for a long coherence time, two hyperfine states in the ground state are preferred as a qubit. The realization that optical tweezers can confine neutral atoms inspired the notion of employing them as qubits.  

% \subsection{Optical Traps}
Optical tweezers are a powerful technique for holding and moving atoms that utilizes the electric dipole interaction force to manipulate quantum objects' external degrees of freedom \cite{samoylenko2020single, nogrette2014single, Lee:16}. Optical tweezers capture and isolate a single atom from a precooled ensemble of atoms without inducing optical transitions, preserving the atom's current state. Optical tweezers are considered a versatile tool that can be achieved using microlens arrays \cite{schlosser2011scalable}, diffractive optical elements, optical standing waves \cite{dammann1971high}, spatial light modulators (SLM) \cite{nogrette2014single} and many more. Holographically generated arrays of tightly focused optical tweezers using SLM are typical and more flexible ways of designing optical tweezers.


% Optical dipole trapping is a simple and powerful technique for holding and steering atoms
% in space [36, 37, 38, 39, 40, 41, 42]. This technique, recently developed to far off-resonant optical trapping (FORT), utilizes the electric dipole interaction force exerted by light to manipulate the external degrees of freedom of quantum objects. From a pre-cooled atom ensemble, focused laser beams can capture and isolate single atoms without inducing optical transitions, so their internal states are well preserved in the electronic ground state up to several seconds, which makes the optically trapped single atoms a promising candidate for storing and processing quantum information. Currently there is a strong interest in using the FORT in engineering scalable quantum platforms [8], [44]-[52], because the manipulation of N single atoms in a synthetic structure of a few µm size is a crucial necessity for the study of quantum computation, quantum simulation, and quantum many-body physics [53]-[56]. Optical dipole microtraps and optical lattices [56] are the well-known tools for atom arrays. In the context of the manipulation and control of individual atoms in an array, optical microtraps are considered to be a versatile tool, having many control parameters. The optical microtraps have been achieved by various methods, including micro lens arrays [57], diffractive optical elements such as Dammann grating [58], spatial light modulators (SLM) [8], [45]-[46], optical standing waves [50], and dynamic light deflectors [45]. With these methods, adiabatic transport
% of atoms in one and two dimensions [47]-[48], atom sorting with a cross junction [50], collisional blockade mechanism [52], controlled collisions for near-deterministic atom loading [59], atom array rotations [46], and single-qubit gate arrays [53] have been demonstrated. These impressive achievements are currently being geared towards a deterministically-loaded high-dimensional arbitrary architecture of N single atoms in which the the internal and external degrees of freedom of the atoms are freely controllable.
% Holographic methods of using a programmable SLM in the Fourier domain have been a
% work horse in the construction and manipulation of various forms of two-dimensional (2D)
% microtrap arrays [8]. The SLM phase pattern generation for Ntr ∼ 2N optical microtraps is
% often performed with iterative Fourier transform algorithms (IFTA).

% % Neutral atom arrays in two- or three-dimensional space may
% % play an important role in quantum information processing
% % (QIP), because of their scalability to a massive number of
% % qubits [1–6]. Currently, arrays of several hundred atoms have
% % been implemented with optically addressable spacings of a
% % few μm [7,8], and this number is expected to increase to a few
% % thousand as laser power permits. These atoms are confined
% % by an array of optical-dipole traps made through various
% % methods including holographic devices [9], diffractive optical
% % elements [10,11], microlens arrays [12], and optical lattices
% % [13]. Ultimately, neutral-atom platforms for QIP may require
% % (i) a significant number of atoms, (ii) a high-dimensional
% % architecture, preferably with an arbitrary lattice geometry,
% % (iii) single-atom loading per site, and (iv) the ability to
% % be individually addressable. However, no existing method
% % satisfies all these requirements. For example, optical lattices
% % can provide a large number of atoms singly loaded per site
% % through theMott insulator transition [13], but they have rather
% % limited geometries and often lack individual addressability;
% % other methods have advantages of arbitrary configurations and
% % site addressability but fail the single-atom


% brief about neutral atom qubit and its importance and benifits

% \subsection{Optical Tweezers}
% trapping of atom and ways to do it\\    


% Trapped neutral atoms have intensively studied as physical implementation of qubits for quantum information processing, along with photons [1], trapped ions [2, 3, 4], and superconductors [5, 6]. As a qubit, a neutral atom, as well as a trapped ion, has a clear advantage that it has a well-defined set of quantum states with sharp linewidth, which makes it a well defined qubit with long coherence time. Besides, the history of controlling atoms, or optical spectroscopy, with many kinds of modulated lights makes it more accessible. Also, neutral atom qubits have high scalability [7], in which they are more favorable than trapped ions, as their trap geometries are easily created by a range of light modulation techniques [8]. On the other hand, achieving high gate fidelity on neutral atoms for quantum computing still remains as a challenge yet. Any two states can be used for a qubit in neutral atoms among many states that have variety of physical characteristics such as linewidth and dipole moments. For long coherence time of qubits, any two hyperfine states in the ground state are favored as a qubit.

In this project, we aim to create optical tweezers for trapping the neutral atoms and then implement the algorithm to generate the array of the trapped atom using dynamic holography. Many research groups have attempted various methods to construct arrays of atomic traps that can hold a single atom or a collection of atoms. The approaches utilized range from the multiple dipole trap beams to the creation of optical lattices and arrays of microtraps using fixed optics. Since the probability of an atom getting trapped in an array is 0.5, the probability of getting a perfect array of the atom is not possible. The inflexibility of these methods in terms of the arrangement of atoms is one of the significant problems. Therefore, we have opted to use the SLM for creating optical tweezers, enabling us to apply dynamic holography and reorganize atoms.


% For literature review of single atom trapping and design of optical tweezers using SLM we have refer to "Single-atom trapping in holographic 2D arrays of microtraps with arbitrary geometries." of Antonie's group\cite{lahaye2013single} and  "Formation and Control of Rydberg Atom Arrays for Quantum Information Processing"\cite{lee2019formation} of Jaewook Ahn's group. We have also refered "Single atom movement with dynamic holographic
% optical tweezers"\cite{samoylenko2020single} and "Design strategies for optimizing holographic optical tweezers setups"\cite{martin2007design}.
% Write in a elucidate way 
% your motivation for considering this specific topic / problem. Also mention / discuss observations / results in this area from previous works / studies / analyses, etc (known as the literature survey).. For example, central force problem is a key area of classical mechanics for which one can consider some standard textbook like the one written by Goldstein.

%For journal you should give name of the article, name 
% of author(s), year of publication, volume, number
% and pages


% Here you need to add references to support your literature survey.
% While referencing remember for books you need to
% provide name of the book, name of author(s), publisher,
% date of publication, pages (if known) and ISBN number.
% For journal you should give name of the article, name 
% of author(s), year of publication, volume, number
% and pages. For reports you can also follow the same.
% Please maintain referencing in proper order, i.e., Ref.[1]
% should come before Ref.[2] and not after Ref.[2] or later, so be careful. Use the option cite\{refname\} (a refname is what
% written inside a backslash sign bibitem\{name\}) with a ``backslash sign" before ``cite'' (without any space) to call a reference.
% Example of correct reference in order: Classical mechanics book by Goldstein \cite{Goldstein:2001} / Famous works of Einstein \cite{Einstein:1916vd,Einstein:1935tc}
% and Dirac \cite{Dirac:1948um} / See IIT Delhi physics \cite{iitd:web}
% (if you need to refer a web link), etc.
% Example of wrong order in referencing: Famous works\footnote[1]{You should read these papers.} (example of putting needed but not-directly connected information in footnote) of Einstein \cite{Einstein:1935tc,Einstein:1916vd}.
